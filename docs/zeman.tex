\documentclass[a4paper, 12pt]{article}

% --- KÓDOVÁNÍ, JAZYK A FONT ---
\usepackage[utf8]{inputenc}
\usepackage[T1]{fontenc}
\usepackage[czech]{babel}
\usepackage{lmodern}      % Kvalitní vektorový font
\usepackage{mathptmx}     % Volitelné: Font Times New Roman (častý požadavek škol)

% --- NASTAVENÍ STRÁNKY ---
\usepackage[margin=2.5cm]{geometry} % Standardní okraje 2.5 cm
\usepackage{setspace}     % Pro nastavení řádkování
\onehalfspacing           % Řádkování 1.5 (standard pro VŠ práce)

% --- TYPOGRAFICKÉ ÚPRAVY ---
\usepackage{microtype}    % Optické vyrovnání okrajů textu
\usepackage{csquotes}     % Správné české uvozovky
\usepackage{parskip}      % Oddělení odstavců mezerou

% --- BALÍČKY PRO GRAFIKU A TABULKY ---
\usepackage{graphicx}
\usepackage{float}        % Pro vynucení pozice obrázků [H]
\usepackage{booktabs}     % Profesionální vodorovné linky v tabulkách
\usepackage{caption}      % Lepší formátování popisků

% --- MATEMATIKA ---
\usepackage{amsmath}

% --- ODKAZY (ČISTÉ, BEZ RÁMEČKŮ) ---
\usepackage[
    hidelinks,            % Odstraní barevné rámečky
    pdfauthor={Jindřich Zeman},
    pdftitle={Filament DryBox Monitor},
    pdfsubject={Technická dokumentace},
    bookmarksnumbered=true
]{hyperref}

% --- FORMÁTOVÁNÍ KÓDU ---
\usepackage{listings}
\usepackage{xcolor}

\definecolor{bg}{rgb}{0.97,0.97,0.97}
\definecolor{comment}{rgb}{0.4,0.4,0.4}
\definecolor{keyword}{rgb}{0.0,0.0,0.5}
\definecolor{string}{rgb}{0.0,0.5,0.0}

\lstdefinestyle{code}{
    backgroundcolor=\color{bg},
    commentstyle=\color{comment}\itshape,
    keywordstyle=\color{keyword}\bfseries,
    stringstyle=\color{string},
    basicstyle=\ttfamily\small,
    breakatwhitespace=false,
    breaklines=true,
    captionpos=b,
    keepspaces=true,
    numbers=left,
    numbersep=10pt,
    showspaces=false,
    showstringspaces=false,
    showtabs=false,
    tabsize=4,
    frame=lines,          % Linky nahoře a dole
    inputencoding=utf8,
    extendedchars=true,
    literate={á}{{\'a}}1 {č}{{\v{c}}}1 {ď}{{\v{d}}}1 {é}{{\'e}}1 {ě}{{\v{e}}}1 
             {í}{{\'i}}1 {ň}{{\v{n}}}1 {ó}{{\'o}}1 {ř}{{\v{r}}}1 {š}{{\v{s}}}1 
             {ť}{{\v{t}}}1 {ú}{{\'u}}1 {ů}{{\r{u}}}1 {ý}{{\'y}}1 {ž}{{\v{z}}}1
}
\lstset{style=code}

% ==========================================
% HLAVNÍ ČÁST DOKUMENTU
% ==========================================
\begin{document}

% --- TITULNÍ STRANA ---
\begin{titlepage}
    \centering
    % Pokud máš název školy, doplň ho sem, jinak nech obecné nebo smaž
    {\Large \textsc{Západoceská univerzita v Plzni}} \\ 
    \vspace{0.5cm}
    {\large Fakulta aplikovaných věd} \\
    \vspace{0.5cm}
    {\large Katedra informatiky a výpočetní techniky}
    
    \vfill % Dynamická mezera
    
    \textbf{\huge Filament DryBox Monitor} \\
    \vspace{1cm}
    {\Large Semestrální práce z předmětu ZPI}
    
    \vfill
    
    \begin{flushleft}
        \large
        \textbf{Autor:} Jindřich Zeman \\
        \textbf{Obor:} Internet věcí (IoT) \\ % Doplň svůj obor
        \textbf{Datum:} \today
    \end{flushleft}
\end{titlepage}

% --- ABSTRAKT ---
\thispagestyle{plain}
\pagenumbering{roman} % Římské číslování pro úvodní listy

\section*{Abstrakt}
Cílem této práce je návrh a realizace monitorovacího zařízení pro sušící boxy (DryBox) využívané při skladování 3D tiskových materiálů. Zařízení řeší problematiku degradace hygroskopických materiálů vlivem vzdušné vlhkosti. Systém je postaven na platformě Raspberry Pi Pico 2 W a využívá architekturu IoT pro sběr, zpracování a distribuci dat. Naměřené hodnoty (teplota, vlhkost, rosný bod) jsou vizualizovány lokálně na OLED displeji a vzdáleně prostřednictvím webového rozhraní a cloudu HiveMQ.

\vspace{1cm}
\textbf{Klíčová slova:} IoT, Raspberry Pi Pico, MQTT, SHT40, Rosný bod, Asynchronní programování.

\newpage

% --- OBSAH ---
\tableofcontents
\newpage

% --- ZAČÁTEK HLAVNÍHO TEXTU ---
\pagenumbering{arabic} % Arabské číslování od 1

% ==========================================
\section{Úvod}
% ==========================================
Aditivní výroba (3D tisk) klade vysoké nároky na kvalitu vstupního materiálu. Většina běžně používaných polymerů, jako jsou PLA, PETG či Nylon, je hygroskopická. To znamená, že pohlcují vzdušnou vlhkost z okolního prostředí. Nasycení filamentu vodou vede při tisku k expanzi páry v trysce, což způsobuje defekty na výtisku a snižuje jeho mechanickou pevnost.

Pro prevenci tohoto jevu se materiály skladují v hermeticky uzavřených boxech s vysoušedlem (tzv. DryBox). Běžné řešení však postrádá "chytrou" funkcionalitu – uživatel nemá přehled o nasycení vysoušedla ani o riziku kondenzace při změně teploty.

Tato práce popisuje vývoj zařízení \textbf{Filament DryBox Monitor}, které poskytuje telemetrii v reálném čase a varuje před kritickými stavy prostředí.

\newpage 

% ==========================================
\section{Analýza a výběr technologií}
% ==========================================
Při návrhu zařízení byl kladen důraz na nízkou spotřebu, přesnost měření a bezpečnost síťové komunikace. Tato kapitola zdůvodňuje výběr jednotlivých komponent a technologií.

\subsection{Hardwarová platforma: Raspberry Pi Pico 2 W}
Pro řízení systému byl zvolen mikrokontrolér Raspberry Pi Pico 2 W. Hlavní důvody pro tuto volbu jsou:
\begin{itemize}
    \item \textbf{Výkon a konektivita:} Čip RP2350 disponuje dostatečným výkonem pro šifrovanou komunikaci (SSL/TLS) a integrovaný Wi-Fi modul (802.11n) eliminuje potřebu externích síťových prvků.
    \item \textbf{Podpora MicroPythonu:} Oficiální podpora tohoto jazyka umožňuje rychlý vývoj a využití moderních programovacích paradigmat (např. \texttt{asyncio}).
    \item \textbf{Cena a dostupnost:} Poměr cena/výkon je v kategorii IoT zařízení bezkonkurenční.
\end{itemize}

\subsection{Senzorika: Sensirion SHT40}
Pro měření environmentálních veličin nebyl použit levný senzor DHT11/22 (který trpí vysokou chybovostí a hysterezí), ale průmyslový standard \textbf{SHT40}.
\begin{itemize}
    \item \textbf{Přesnost:} Typická odchylka $\pm 1.8\,\% RH$ je klíčová pro přesný výpočet rosného bodu.
    \item \textbf{Rozhraní:} Komunikace přes sběrnici I$^2$C je robustnější než proprietární protokoly levných čidel.
\end{itemize}

\subsection{Komunikační protokol: MQTT}
Pro přenos dat do cloudu byl zvolen protokol \textbf{MQTT} (Message Queuing Telemetry Transport) místo běžného HTTP.
\begin{itemize}
    \item \textbf{Efektivita:} MQTT je navržen pro sítě s vysokou latencí a nízkou propustností. Oproti HTTP má výrazně menší datovou hlavičku (overhead).
    \item \textbf{Model Publish/Subscribe:} Zařízení nemusí čekat na odezvu serveru pro každou zprávu, což šetří čas procesoru i energii.
    \item \textbf{Bezpečnost:} Cloudová implementace (HiveMQ) vyžaduje šifrování SSL/TLS, což zajišťuje integritu dat.
\end{itemize}

\newpage

% ==========================================
\section{Návrh a realizace}
% ==========================================
Tato kapitola popisuje technické řešení, zapojení a softwarovou architekturu.

\subsection{Schéma zapojení}
Systém využívá dvě oddělené I$^2$C sběrnice mikrokontroléru, aby se předešlo konfliktům adres a rušení mezi displejem a senzorem.

\begin{table}[H]
\centering
\caption{Zapojení komponent k Raspberry Pi Pico}
\label{tab:wiring}
\begin{tabular}{@{}llcc@{}}
\toprule
\textbf{Periferie} & \textbf{Pin MCU} & \textbf{Funkce} & \textbf{Sběrnice} \\ \midrule
\textbf{OLED Displej} & GP0 & SDA & I2C0 \\
(Řadič SH1106)       & GP1 & SCL & I2C0 \\ \midrule
\textbf{Senzor SHT40} & GP2 & SDA & I2C1 \\
                     & GP3 & SCL & I2C1 \\ \bottomrule
\end{tabular}
\end{table}

\subsection{Softwarová architektura}
Aplikace není psána jako sekvenční smyčka (což by blokovalo běh programu při síťové komunikaci), ale využívá kooperativní multitasking pomocí knihovny \texttt{uasyncio}.

Program se skládá ze tří asynchronních úloh (\textit{Tasks}), které běží kvazi-paralelně:
\begin{enumerate}
    \item \textbf{Task Měření a UI:} S periodou 3 sekund vyčítá data, počítá rosný bod a překresluje OLED displej.
    \item \textbf{Task Webserver:} Běží na pozadí a čeká na HTTP požadavky na portu 80. Poskytuje "Dark Mode" dashboard.
    \item \textbf{Task Cloud:} S periodou 30 sekund odesílá data přes MQTT. Tento proces zahrnuje řízení paměti (Garbage Collection), aby nedošlo k přetečení RAM při náročném SSL handshaku.
\end{enumerate}

\subsection{Výpočet rosného bodu}
Rosný bod ($T_{dp}$) je teplota, při které je vzduch maximálně nasycen vodními parami. Pokud teplota klesne pod tuto hodnotu, dochází ke kondenzaci. Pro výpočet byl implementován \textbf{Magnusův vzorec}:

\begin{equation}
    T_{dp} = \frac{237,7 \cdot \ln\left(\frac{RH}{100} \cdot e^{\frac{17,27 \cdot T}{237,7 + T}}\right)}{17,27 - \ln\left(\frac{RH}{100} \cdot e^{\frac{17,27 \cdot T}{237,7 + T}}\right)}
\end{equation}

Kde $T$ je teplota ve stupních Celsia a $RH$ je relativní vlhkost v procentech.

\newpage

% ==========================================
\section{Zprovoznění a konfigurace}
% ==========================================
Zařízení je navrženo jako "Plug \& Play", avšak vyžaduje úvodní konfiguraci síťových parametrů.

\subsection{Instalace firmwaru}
\begin{enumerate}
    \item Připojte Raspberry Pi Pico 2 W k PC při stisknutém tlačítku BOOTSEL.
    \item Nahrajte aktuální verzi MicroPython firmwaru (soubor \texttt{.uf2}).
    \item Do kořenového adresáře nahrajte soubory projektu (\texttt{main.py}, \texttt{simple.py}, \texttt{sh1106.py}, \texttt{sht40.py}).
\end{enumerate}

\subsection{Bezpečnostní konfigurace}
Přihlašovací údaje nejsou součástí zdrojového kódu. Uživatel musí vytvořit soubor \texttt{secrets.py} dle následujícího schématu. Tento soubor je následně vyloučen ze verzování (git ignore) pro zachování bezpečnosti.

\begin{lstlisting}[language=Python, caption=Struktura konfiguračního souboru secrets.py]
secrets = {
   'ssid': 'NAZEV_WIFI_SITE',
   'password': 'HESLO_WIFI',
   
   # Nastavení HiveMQ Cloudu
   'mqtt_broker': 'adresa_clusteru.hivemq.cloud',
   'mqtt_port': 8883,
   'mqtt_user': 'uzivatel',
   'mqtt_pass': 'heslo',
}
\end{lstlisting}

\newpage

% ==========================================
\section{Závěr}
% ==========================================
V rámci této práce bylo navrženo a realizováno IoT zařízení pro monitorování podmínek skladování 3D tiskových materiálů. 

Podařilo se splnit všechny stanovené cíle:
\begin{itemize}
    \item Zařízení měří teplotu a vlhkost s využitím profesionálního senzoru.
    \item Výpočet rosného bodu poskytuje uživateli relevantní informaci o riziku navlhnutí materiálu.
    \item Implementace MQTT s SSL šifrováním zajišťuje bezpečnou integraci do chytré domácnosti.
    \item Optimalizace paměti pomocí Garbage Collectoru vyřešila problémy se stabilitou na mikrokontroléru.
\end{itemize}

Výsledné zařízení představuje cenově dostupné, ale technicky pokročilé řešení problému degradace filamentů.

\end{document}